% \documentclass[a4paper,10pt]{article}
\documentclass[twocolumn,a4paper,10pt]{article}
\usepackage{geometry}
\usepackage[T1]{fontenc}
\usepackage[utf8]{inputenc}
\usepackage{lmodern}
\usepackage[english]{babel}
\usepackage{url}
\usepackage{hyperref}
\usepackage{subcaption}
\usepackage{graphicx}
\usepackage{amsmath}

\geometry{a4paper,
 left=15mm,
 right=15mm,
 top=20mm,
 bottom=20mm,
}


% Training subcaption package to comply with
% IEEE standards. We can ignore the warning
% generated by caption.sty which is due to 
% the redefinition of \@makecaption
\DeclareCaptionLabelSeparator{periodspace}{.\quad}
\captionsetup{font=footnotesize,labelsep=periodspace,singlelinecheck=false}
\captionsetup[sub]{font=footnotesize,singlelinecheck=true}


\title{Machine Learning Homework 3:\\Unsupervised Learning and Dimensionality Reduction}
\author{Nicolas Six}







\makeatother

\begin{document}
\maketitle \tableofcontents{}

%%%%%%%%%%%%%%%%%%%%%%%%%%%%%%%%%%%%%%%%%%%%%%%%%%%%%%%%%%%%%%%%%%%%%%%%%%%%%%%%%%%%%%%%%%%%%%%%%

\section{Introduction}

\paragraph{}

In this work we will focus on dimension reduction and clustering.
We will try here to show how clustering and dimension reduction behave
in different situations. To do this we will first work on direct application
of clustering in the particular case where we went to fund back known
cluster. We will then do the same study but with an additional step
of dimension reduction before clustering. After that we will explore
how we can use dimension reduction algorithm as a first step in training
a neural network as well as consider the option of adding a clustering
layer to generate new features for the neural network.

\subsection{Datasets}

\label{intro_dataset} 

\paragraph{}

For this work we focused on two different dataset. The first is the
well known Iris dataset, this dataset describe four different parameters
of three different Iris variety and contain 151 different entries.
We chose this dataset because of the low dimension space that it cover,
allowing us to plot easily understandable 2D figures to show the results
of clustering and dimension reduction algorithms. Figure \ref{fig:iris_org}
show the repartition of the classes on the dataset. You can see that
the classes are already well defined and we can so suppose that clustering
algorithm will perform well.

\begin{figure}
\centering \begin{subfigure}[t]{0.49\columnwidth} \centering
\includegraphics[width=1.1\linewidth]{{../plotter/graphs/org_sepal_length_sepal_width_label}.png}
\caption{Classes repartition according to sepal length and width}
\label{fig:iris_org_sep} \end{subfigure} \begin{subfigure}[t]{0.49\columnwidth}
\centering \includegraphics[width=1.1\linewidth]{{../plotter/graphs/org_petal_length_petal_width_label}.png}
\caption{Classes repartition according to petal length and width}
\label{fig:iris_org_pet} \end{subfigure} \caption{Repartition of the classes on Iris dataset according to different
parameters}
\label{fig:iris_org} 
\end{figure}

The second dataset is here again the player recognition dataset on
Starcraft II. For a more in depth description please refer to Homework
1 and 2. This dataset is again very challenging here, as it as 200
classes visualization will be hard and it's 72 dimensions wont help
to represent it on paper. In addition with only 3035 examples cluster
are very small in average, with some having 4 elements while others
have more than 50. As you can see on Figure \ref{fig:sc_org} the
plots are not readable, even when we select dimension which are known
to easily differentiate players, such as the action per minutes (APM)
or the race used. We also explored the possibility to add a third
dimension, but it didn't improve the visualization here. In addition
the different feature of this dataset are on very different intervals,
going from 0 to 3 to 0 to 7000. As we know that such characteristic
doesn't help clustering algorithm we choose to normalize all the different
features.

\begin{figure}
\centering \begin{subfigure}[t]{0.49\columnwidth} \centering
\includegraphics[width=1.1\linewidth]{{../plotter/graphs/org_avg_micro_apm_max_ap5s_label}.png}
\caption{Classes repartition according to avg APM and max number of action
per 5 seconds}
\label{fig:sc_org_apm} \end{subfigure} \begin{subfigure}[t]{0.49\columnwidth}
\centering \includegraphics[width=1.1\linewidth]{{../plotter/graphs/org_race_avg_micro_apm_label}.png}
\caption{Classes repartition according to avg APM and race used by the player}
\label{fig:sc_org_race} \end{subfigure} \caption{Repartition of the classes on Starcraft dataset according to different
parameters}
\label{fig:sc_org} 
\end{figure}

%%%%%%%%%%%%%%%%%%%%%%%%%%%%%%%%%%%%%%%%%%%%%%%%%%%%%%%%%%%%%%%%%%%%%%%%%%%%%%%%%%%%%%%%%%%%%%%%%

\section{Clustering}

\paragraph{}

Applying clustering like K-means or Expectation Maximization (EM)
to our two different dataset give good results.

For the Iris dataset both clustering algorithms perform really well.
As you can see on Figure \ref{fig:iris_kmeans}, kmeans get here a
near perfect classification result. The typical potatoid shape of
the kmeans clusters is easy to spot on Figure \ref{fig:iris_kmeans_pet},
but this particular bias of kmeans is also the reason why it make
a few mistakes that are easy to spot on Figure \ref{fig:iris_kmeans_sep}.
EM on it side perform even better than kmeans here. Few mistakes can
be spotted on the edges between blue and white area of Figure \ref{fig:iris_EM_pet},
which is not surprising as the two original cluster were very close
on every dimensions. In addtion we can note that given it's nature
EM let more the blue and white cluster to mix together on Figure \ref{fig:iris_EM_sep},
this is due to the large space between point in this part of the cluster.

Such good results are not surprising for Iris dataset as it's really
a toy dataset in terms of complexity, this was desired here to show
how the algorithm behave.

\begin{figure}
\centering \begin{subfigure}[t]{0.49\columnwidth} \centering
\includegraphics[width=1.1\linewidth]{{../plotter/graphs/kmeans_sepal_length_sepal_width_label}.png}
\caption{Classes repartition according to sepal length and width}
\label{fig:iris_kmeans_sep} \end{subfigure} \begin{subfigure}[t]{0.49\columnwidth}
\centering \includegraphics[width=1.1\linewidth]{{../plotter/graphs/kmeans_petal_length_petal_width_label}.png}
\caption{Classes repartition according to petal length and width}
\label{fig:iris_kmeans_pet} \end{subfigure} \caption{Repartition of the K-Means clustering results on Iris dataset according
to different parameters}
\label{fig:iris_kmeans} 
\end{figure}

\begin{figure}
\centering \begin{subfigure}[t]{0.49\columnwidth} \centering
\includegraphics[width=1.1\linewidth]{{../plotter/graphs/EM_sepal_length_sepal_width_label}.png}
\caption{Classes repartition according to sepal length and width}
\label{fig:iris_EM_sep} \end{subfigure} \begin{subfigure}[t]{0.49\columnwidth}
\centering \includegraphics[width=1.1\linewidth]{{../plotter/graphs/EM_petal_length_petal_width_label}.png}
\caption{Classes repartition according to petal length and width}
\label{fig:iris_EM_pet} \end{subfigure} \caption{Repartition of the EM clustering results on Iris dataset according
to different parameters}
\label{fig:iris_EM} 
\end{figure}

The Starcraft dataset is tricky to represent and analyses for the
same reason as described in section \ref{intro_dataset}. However,
if in general both clustering did not get good results we can note
that the outliers are generally classified together, which is the
right things to do according to Figure \ref{fig:sc_org}. This is
easy to spot on Figures \ref{fig:sc_kmeans_sep} and \ref{fig:sc_EM_sep}.
Such a comportment is not surprising as both clustering algorithms
try to group similar features together, thus outliers are easily grouped
while at the same time easy to spot when looking at the plots. The
fact that those outliers are in the same class here allow us to suppose
that they are close in others dimensions, as is they were not K-Means
and EM will have chosen a different classes for each point or found
link on other dimension to give result such as the one of Figure \ref{fig:iris_EM_sep}.

\begin{figure}
\centering \begin{subfigure}[t]{0.49\columnwidth} \centering
\includegraphics[width=1.1\linewidth]{{../plotter/graphs/kmeans_avg_micro_apm_max_ap5s_label}.png}
\caption{Classes repartition according to avg APM and max number of action
per 5 seconds}
\label{fig:sc_kmeans_sep} \end{subfigure} \begin{subfigure}[t]{0.49\columnwidth}
\centering \includegraphics[width=1.1\linewidth]{{../plotter/graphs/kmeans_race_avg_micro_apm_label}.png}
\caption{Classes repartition according to avg APM and race used by the player}
\label{fig:sc_kmeans_pet} \end{subfigure} \caption{Repartition of the K-Means clustering results on Starcraft dataset
according to different parameters}
\label{fig:sc_kmeans} 
\end{figure}

\begin{figure}
\centering \begin{subfigure}[t]{0.49\columnwidth} \centering
\includegraphics[width=1.1\linewidth]{{../plotter/graphs/EM_avg_micro_apm_max_ap5s_label}.png}
\caption{Classes repartition according to avg APM and max number of action
per 5 seconds}
\label{fig:sc_EM_sep} \end{subfigure} \begin{subfigure}[t]{0.49\columnwidth}
\centering \includegraphics[width=1.1\linewidth]{{../plotter/graphs/EM_race_avg_micro_apm_label}.png}
\caption{Classes repartition according to avg APM and race used by the player}
\label{fig:sc_EM_pet} \end{subfigure} \caption{Repartition of the EM clustering results on Starcraft dataset according
to different parameters}
\label{fig:sc_EM} 
\end{figure}

To the light of those results we can say that clustering algorithm
are really dependent on the statistical repartition of the data. Even
if data are normalized, density is not constant and induce bias in
the clustering algorithm. An other bias came from the redundancy of
the data. For Iris, it's not really the case as it has very few dimensions
which were selected by an expert and have thus their part of domain
knowledge. However, for Starcraft we can assume that a lot of information
is redundant as no particular filter were applied, and correlation
between features such as the average APM and the max APM on 5 seconds
are trivial. This redundancy is equivalent to give more weight to
some features, which can be very harmful to clustering if this features
are not relevant. In the next section we will thus try different algorithms
which do such things, without adding any domain knowledge.

%%%%%%%%%%%%%%%%%%%%%%%%%%%%%%%%%%%%%%%%%%%%%%%%%%%%%%%%%%%%%%%%%%%%%%%%%%%%%%%%%%%%%%%%%%%%%%%%%

\section{Dimension reduction}

\label{dr} 

\paragraph{}

Reduce the number of dimension is often crucial in machine learning,
particularly because the curse of dimensions. A lot of different algorithms
exist to try to reduce the number of dimension to a set of meaningful
one. But meaningful does not have any meaning on its own and it's
probably why they are so many different ways to do it.

Here we will explore the results of the dimension reduction algorithms
seen during the lessons: PCA, ICA, Random projections and LDA. The
later is a bit particular as it make the projection knowing what you
want to find, it's also why chose it, to show how giving such a crucial
information change the results.

\subsection{PCA}

\paragraph{}

An interesting possibility with PCA is that you can easily score the
dimensions chosen thanks to the associated eigenvalue. Plots representing
those values for both datasets can be fund on Figure \ref{fig:pca_score}.
On both case the eigenvalue follow an exponentially decreasing shape
with the number of dimensions. Thus you can see that on Iris, according
to PCA, one dimension already give a good description of the dataset
while ten dimensions is still short to represent the Starcraft dataset.

\begin{figure}
\centering \begin{subfigure}[t]{0.49\columnwidth} \centering
\includegraphics[width=1.1\linewidth]{{../plotter/graphs/PCA_dimension_importance_4}.png}
\caption{Iris dataset}
\label{fig:pca_iris_score} \end{subfigure} \begin{subfigure}[t]{0.49\columnwidth}
\centering \includegraphics[width=1.1\linewidth]{{../plotter/graphs/PCA_dimension_importance_70}.png}
\caption{Starcraft dataset}
\label{fig:pca_sc_score} \end{subfigure} \caption{Evolution of the score according to the number of dimension for both
dataset}
\label{fig:pca_score} 
\end{figure}

\begin{figure}
\centering \begin{subfigure}[t]{0.49\columnwidth} \centering
\includegraphics[width=1.1\linewidth]{{../plotter/graphs/pca_iris_0_1_label}.png}
\caption{Iris dataset}
\label{fig:pca_iris} \end{subfigure} \begin{subfigure}[t]{0.49\columnwidth}
\centering \includegraphics[width=1.1\linewidth]{{../plotter/graphs/pca_sc_0_1_label}.png}
\caption{Starcraft dataset}
\label{fig:pca_sc} \end{subfigure} \caption{Representation of the classes after projection in the best plan fund
by PCA}
\label{fig:pca} 
\end{figure}

As you can see on Figure \ref{fig:pca}, the projections does it job.
In the Iris case, the classes really appear on different clusters.
We can also understand here why axes 0 as a better score than axes
1, as here the data nearly appear like group of vertical line, which
is great given that PCA didn't know the labels, but still managed
to nearly split them on the first dimension.

For the Starcraft dataset the things are a bit different, if it's
sure that the data are more widely spread in Figure \ref{fig:pca_sc}
than it was before a PCA, the 200 classes still appear as noise. This
is not surprising given that we try here to represent in two dimensions
data that PCA say it can represent correctly in ten to twenty dimensions.
It's hard to know if it's our brain that try to fit pattern or if
they are really pattern here, but we can see that same colors are
generally on similar area, which show that PCA find some consistent
way to reduce the dimensionality, but that two dimensions wasn't enough.

\paragraph{}

In a clustering point of view, Iris is still an easy case. K-Means
get here worst result than on the full dataset because it try to split
the two cluster on the right in a horizontal way. This comportment
can be countered by a different norm or data normalization, but doing
so can be called supervised learning, which is not our goal here.
EM get here near perfect results, probably taking more advantage of
the hight density in the center of the clusters.

Starcraft suffer again from the lake of dimension, both clustering
algorithms try to find cluster that does not exist. While this gave
us beautiful plot of a large potatoid full of small colorful potatoid,
but we did not get back the players we were looking for. Which is
not really surprising when you know how the players are split (Figure
\ref{fig:pca_sc}).

\paragraph{}

In conclusion, PCA is fast and efficiently reduce the number of dimensions
while giving us nice values to help us to chose how much dimension
we want.

\subsection{ICA}

\label{dr_ica} 

\paragraph{}

ICA look for independent underling variable and is so very used in
signal processing when different signals are mixed together. In our
case, it's probably possible to find underling variable to our data,
the principal one being the label, but on our two cases it's hard
to find other examples. This does not mean that ICA will not perform
well here but that it's probably not the best algorithm we can use.

On the Iris dataset, as you can see on Figure \ref{fig:ica_iris},
the result are very similar than the ones of PCA (Figure \ref{fig:pca_iris}).
If you perform a 180 degree rotation of the plot you will see that
they are nearly identical. Thus, we can apply the same conclusion
than for PCA here. But it's important to note that the scale are different,
here the ratio of the range of axis 0 over the one of axis 1 is about
0.7, while it was 2.4 in PCA, explaining the clustering difference.

On Starcraft, The potatoid shape is quite different from PCA but it
is not enough to split the different players apart, as you can see
on Figure \ref{fig:ica_sc}. Still the result look better than for
PCA, as color spot are easier to find.

\begin{figure}
\centering \begin{subfigure}[t]{0.49\columnwidth} \centering
\includegraphics[width=1.1\linewidth]{{../plotter/graphs/ica_iris_0_1_label}.png}
\caption{Iris dataset}
\label{fig:ica_iris} \end{subfigure} \begin{subfigure}[t]{0.49\columnwidth}
\centering \includegraphics[width=1.1\linewidth]{{../plotter/graphs/ica_sc_0_1_label}.png}
\caption{Starcraft dataset}
\label{fig:ica_sc} \end{subfigure} \caption{Representation of the classes after projection in the best plan fund
by ICA}
\label{fig:ica} 
\end{figure}

\paragraph{}

On a clustering point of view, due to change of proportion both K-Means
and EM wrongly split the left cluster and preferred to split them
horizontally. In addition K-means managed to get wrong one point of
the right cluster. As described for PCA, axes 0 is here the one carrying
the more information, reducing it's amplitude regarding to the one
of axes 1, give far more importance to this last axis in both clustering
experiments. Again one may say that this problem is easy to overcome,
but doing it is not our goal here as it tend to become supervised
learning with its own problem such as overfitting.

Starcraft is still difficult here for clustering algorithms. The result
are very similar to the ones of PCA. As the classes are really mixed
together, clustering only manage to group some sort of similar player
but didn't manage to find the real player.

\paragraph{}

In conclusion, ICA gave us very similar results than PCA, probably
because our data were too difficult, or too easy, to differentiate
the two algorithms, but also because none of the datasets has trivial
underling variables.

\subsection{Randomized projections}

\paragraph{}

Random projection is a very time efficient way to reduce dimension.
The fact that it is random driven let us expect worst results than
the other algorithms explored here.

On Iris most of the random projection we tried gave average results,
not better than a plan in the original nor really worst. As you can
see on Figure \ref{fig:rnd_iris}, the global shape is familiar as
very close to the one get for PCA and ICA, with one cluster far from
the two others which are really close but with very few common part.
This projection is far from optimal in a clustering point of view,
but the cluster are still there.

The results on Starcraft are more interesting. Not that the player
are well grouped together, but this projection managed to mix feature
in a way that allow out-layers to stand out of the global mass. Thus
we cannot expect clustering algorithm to get every classes right but
they might fund so of them, which were not rely the case of both PCA
and ICA.

\begin{figure}
\centering \begin{subfigure}[t]{0.49\columnwidth} \centering
\includegraphics[width=1.1\linewidth]{{../plotter/graphs/rnd_iris_0_1_label}.png}
\caption{Iris dataset}
\label{fig:rnd_iris} \end{subfigure} \begin{subfigure}[t]{0.49\columnwidth}
\centering \includegraphics[width=1.1\linewidth]{{../plotter/graphs/rnd_sc_0_1_label}.png}
\caption{Starcraft dataset}
\label{fig:rnd_sc} \end{subfigure} \caption{Representation of the classes after projection in a plan by random
projection}
\label{fig:rnd} 
\end{figure}

\paragraph{}

Clustering on random projection mostly gave bad results on Iris. For
example, EM chose here to group the two bottom cluster together and
to split the top one. While K-Means cut the bottom group in two with
a horizontal line, one of the worst chose if you want to find the
original labels.

On Starcraft, the clustering algorithm unsurprisingly gave bad results.
Some outliers were correctly classified but were also most of the
time split in multiple clusters. The central mass was as previously
too diffuse to get any thing right out of it. As previously we can
say that 2 dimensions are not enough to represent those data. Particularly
with random projection, which may need more dimensions to conserve
the same amount of information than PCA and ICA.

\paragraph{}

Thus, random projection does not do a great job to apply clustering
algorithm, but still keep the global shape of the data and can be
useful for quick visualization purpose or as reference to evaluate
the performance of other dimension reduction algorithms.

\subsection{LDA}

\paragraph{}

LDA is very different from the others dimension reduction algorithm
explored in this report, as it is the only one to include the labels
to make the best projection. This information make the whole clustering
more supervised than other thing. Thus, LDA does not give the same
information than other algorithms, as this one is biased toward what
we are looking for. This bias is interesting here to see how this
information change the results, but usually clustering is used when
you do not know what you are looking for, case where LDA is useless.

The projection of Iris into two dimensions displayed on Figure \ref{fig:lda_iris}
show particularly well split clusters. Two classes are still very
close to each other, but the original data want that. We can, here,
also look at the range ratio, but it will not be useful as axes 0
already give most of the information needed to correctly classify
every thing, as long as this ration stay near 1, we can expect good
clustering results later. So LDA project here the Iris dataset into
nearly one dimension, showing that with only a linear combination
of the variable we have enough information to find all the classes
of this dataset.

The best projection fund by LDA for Starcraft is a bit different in
term of shape from the one seen previously. Me must also note than
players seems to be more close to each others than before, but players
are still mixed together. So even when including what we are looking
for, we are not able to linearly combine dimensions to project data
to a plan where clustering algorithms will be able to manage them.
Knowing that it is possible to get 90\% validation accuracy on this
problem we can deduce that, ether we need more dimension or linear
combination is not sufficient here.

\begin{figure}
\centering \begin{subfigure}[t]{0.49\columnwidth} \centering
\includegraphics[width=1.1\linewidth]{{../plotter/graphs/lda_iris_0_1_label}.png}
\caption{Iris dataset}
\label{fig:lda_iris} \end{subfigure} \begin{subfigure}[t]{0.49\columnwidth}
\centering \includegraphics[width=1.1\linewidth]{{../plotter/graphs/lda_sc_0_1_label}.png}
\caption{Starcraft dataset}
\label{fig:lda_sc} \end{subfigure} \caption{Representation of the classes after projection in the best plan fund
by LDA}
\label{fig:lda} 
\end{figure}

\paragraph{}

As already stated for Iris, the clustering after this projection is
stretch forward. Both EM and K-Means does near perfect classification
here in terms of original label. Both of them with some errors in
the frontier of the two clusters, but in a smaller way than what we
have seen before. They do not get confused by the second dimension
probably because it range on a smaller interval than the first one.

For Starcraft, it's difficult to say if the result are better or worse
than before. We can probably say that they are as worse as in all
the previous cases. If you can see, on Figure \ref{fig:lda_sc}, that,
globally, all the games of a player tend to be on the same region,
they are still too mixed with other players to allow clustering algorithm
to find them.

\paragraph{}

In conclusion LDA does a great job on Iris but doesn't do miracles
on Starcraft. In addition for this little improvement with regard
to other algorithm such as PCA and ICA, LDA come with a large cost,
the one of the labels. It is so not used in the same cases than the
previous ones. While the tree others algorithms are very useful when
you look for something in the data but not knowing exactly what, LDA
is useful when you know what you are looking for and just want to
reduce the dimensionality before running classification algorithm
for example.

\subsection{Conclusion}

%execution time of EM on starcraft + Iris
% +3D

\paragraph{}

In this section we have seen that applying dimension reduction before
clustering can be very efficient to compress the data while keeping
them as meaningful as possible. This was really useful here for visualization
purpose, which is not negligible when the main question is describe
what you see, but can also be very useful to limit the effect of the
curse of dimensionality.

Run time were not really discussed before, mainly because most of
them are not significant. All the dimension reduction algorithm does
there work in less than a few seconds, it's also the same for K-means,
but EM is far slower taking about a minute to complete all the 30
runs asked (all of them being initialized by running a K-Means). Thus
if EM generally perform better than K-Means, their timing difference
probably explain why K-Means is still very popular.

We chosen to keep Starcraft dataset here to have an example of a case
were both clustering and dimension reduction were not capable of producing
exploitable data. We thus wanted to explore what kind of results we
can expect from such high dimension data. %This exploration is important, when you explore a new dataset, figuring out if you are more in the Iris case than Starcraft case quickly can save a lot of time.

As a comparison, we also tried to apply our network from homework
1 to Starcraft with a layer of only two neurones to force it to do
a dimension reduction. On all the possibility tested, the best validation
accuracy was lower than 7\%. This result is an other indication that
lot of information are lost by reducing the dimension to two.

To improve this report we also tried to display Starcraft on three
dimensions plots, but those plots were even less readable than the
two dimensions one. In addition if human eyes can distinguish millions
of colors, we did not fund a good way to make all the players perfectly
distinguishable on the figures, while this is not crucial for our
conclusion, we are sorry for the inconvenience.

We also apologies for the lack of figures in this section, but due
to space constraint we have to do some sacrifice. Thus the plots of
the different cluster were not included here as easy to imagine from
the projections.

%%%%%%%%%%%%%%%%%%%%%%%%%%%%%%%%%%%%%%%%%%%%%%%%%%%%%%%%%%%%%%%%%%%%%%%%%%%%%%%%%%%%%%%%%%%%%%%%%

\section{Training a Neural Network after dimension reduction}

\label{nn} 

\paragraph{}

Reducing dimension before trying to learn a classifier such as a neural
network (NN) is a good solution to limit the effect of the curse of
dimensions. One may expect that reducing the number of dimension will
reduce the convergence time of a neural network, but this may not
be true as compressing the information may make it harder for a NN
to find useful way to use it.

To properly asses that question, among others, we will use here the
same network structure from Homework 1. In this structure, we set
most of the parameters explored in this previous work. For example,
we used batch normalization in all our experiment, as well as the
sigmoid activation function and the rmsprop optimizer from Keras.
Thus, the reader can easily compare the results with the one presented
previously.

We will also stick to the same graphical convention as previously:
the hard line represent the median value while the hue area go from
the minimum to the maximum.

The same convention on the description of the NN structure will also
hold. $(n)$ describe a perceptron with one hidden layer of $n$ neurones
and $(a,b)$ a perceptron with two hidden layers of $a$ and $b$
neurones respectively.

\paragraph{}

If the part of the structure of the NN stay the same than in Homework
1, we still explored here a different set of parameters. The most
important among them are number of features, the dimension reduction
algorithm and the shape of the hidden layers. This last parameters
cover the same range as in Homework 1, to keep comparable results.
The dimension reduction was first fit on the train dataset and then
applied to both train and test data. Repartition between train and
test reminded the same than in assignment one, please refer to it
for more information.

\paragraph{}

Before going more in depth, algorithm by algorithm, we will draw here
some global trends.

As you can see on Figure \ref{fig:per_all_rm}, the different reduction
methods does not gave the same results. However, we must note that,
apart from the particular case of LDA, the median result of all the
algorithm converge toward the same accuracy. It also worth mentioning
that ICA gave some hard time to the NN with first improvements appearing
far later than others methods.

It was also easy to predict that the more feature you give the better
the result. Such a comportment can be seen on Figure \ref{fig:per_all_nf},
where you can see that there is a clear progression from 5 to 20 features
but that the evolution is then far smaller. This is particularly truth
for the maximum, which have nearly all the same values. Thus, as a
first impression we can say that 20 features is enough for a NN to
learn how to classify the different players. We must also note that
adding new feature does not augment the number of epoch needed by
the NN to converge. This is probably because with more features it's
easier for gradient descant to find one to focus on.

\begin{figure}
\centering \begin{subfigure}[t]{0.7\columnwidth} \centering
\includegraphics[width=1.1\linewidth]{{../plotter/graphs/all_epoch_val_categorical_accuracy_number_of_feature}.png}
\caption{Effect of the number of feature}
\label{fig:per_all_nf} \end{subfigure} \begin{subfigure}[t]{0.7\columnwidth}
\centering \includegraphics[width=1.1\linewidth]{{../plotter/graphs/all_epoch_val_categorical_accuracy_reduction_method}.png}
\caption{Effect of the reduction method used}
\label{fig:per_all_rm} \end{subfigure} \caption{Validation accuracy according to the epoch for all the different combination}
\label{fig:per_all} 
\end{figure}

\subsection{PCA}

\paragraph{}

PCA is the best performer here, after LDA. As you can see on Figure
\ref{fig:per_pca_nf}, the global variation observed previously with
the number features perfectly fit here. It is also hard not to notice
the pick near epoch 400 for 40 features. By selecting only a subset
of layers we can see that, as displayed on Figure \ref{fig:per_pca_nf_3525},
this pick is created by layers $(35)$ and $(25)$, that are able
to take advantage of the extra features to improve their accuracy.
The interesting fact here is that this advantage is quickly transformed
into overfitting. Overfitting also occur for 35 features, with accuracy
in that case never exceeding the one get with 30 features going as
low as the one with only 10 features.

Adding new input features rise the number of parameters in the network,
which also rise it chance to overfit. This is also know as the curse
of dimensions. The lack of different examples here probably forbid
the NN to properly integrate the new features letting it overfit on
the train set.

In terms of layers the results are here very similar to the one of
the first homework. The different numbers of features given by PCA
does not significantly improve the result of a given layer, and it
is the same the other way around. It is important here to empathize
that, if the number of features does not change the convergence speed,
the size of the layers do. The validation accuracy of small layers
is rising slower than the one of layers with more neurones. This is
probably because, having more weights, the bigger layers are easier
to fit into a local optimum. So if you want to approximate a complex
function it's easier with lot of simple one than few complex one,
even if the later is probably more reliable.

\paragraph{}

In conclusion PCA allow a perceptron to learn good results on fewer
features. But, those results are not as good as the one obtained on
the original dataset and training is slower in terms of epochs. But
we are here in a case with enough examples to manage the curse of
dimension, in a case where we are not, PCA look like a good choice
to reduce its effects.

\begin{figure}
\centering \begin{subfigure}[t]{0.7\columnwidth} \centering
\includegraphics[width=1.1\linewidth]{{../plotter/graphs/PCA_epoch_val_categorical_accuracy_number_of_feature}.png}
\caption{Effect of the number of features for all tested layers}
\label{fig:per_pca_nf} \end{subfigure} \begin{subfigure}[t]{0.7\columnwidth}
\centering \includegraphics[width=1.1\linewidth]{{../plotter/graphs/PCA_epoch_val_categorical_accuracy_number_of_feature_35_25}.png}
\caption{Effect of the number of features for $(35)$ and $(25)$ layers}
\label{fig:per_pca_nf_3525} \end{subfigure} \caption{Validation accuracy according to the epoch with PCA}
\label{fig:per_pca} 
\end{figure}

\subsection{ICA}

\label{nn_ica} 

\paragraph{}

The first things you probably note on Figure \ref{fig:per_all_rm},
is that ICA give features that make the perceptron slower to converge.
This behavior is common to all the combination explored here, as you
can see on Figure \ref{fig:per_ica}. Thus we suppose that the way
ICA split features is not an easy one to understand for a neural network,
even if the valuable information is still present as it finally get
a similar median accuracy than both PCA and random projections.

We can also notice that as with PCA, having 35 or 40 features give
overfitting, for every layers tried here. The explanation given on
the PCA case is still relevant on this case.% We can even go further here and add that this overfitting is worst when the hidden layer is bigger.

In terms of accuracy according to the number of feature, ICA give
here very similar results than PCA once convergence is reached. Although
the maximum validation accuracy reached by ICA is smaller than the
one on PCA, while the two were run with the same set of parameters.

\begin{figure}
\centering \begin{subfigure}[t]{0.7\columnwidth} \centering
\includegraphics[width=1.1\linewidth]{{../plotter/graphs/ICA_epoch_val_categorical_accuracy_number_of_feature}.png}
\caption{Effect of the number of features for all tested layers}
\label{fig:per_ica_nf} \end{subfigure} \begin{subfigure}[t]{0.7\columnwidth}
\centering \includegraphics[width=1.1\linewidth]{{../plotter/graphs/ICA_epoch_val_categorical_accuracy_number_of_feature_35_25}.png}
\caption{Effect of the number of features for $(35)$ and $(25)$ layers}
\label{fig:per_ica_nf_3525} \end{subfigure} \caption{Validation accuracy according to the epoch with ICA}
\label{fig:per_ica} 
\end{figure}

\paragraph{}

Thus, ICA give good median results after convergence, but those results
are not as good as PCA and convergence need a lot more epochs than
for this last algorithm. So, we consider that ICA is not really relevant
in our case, probably because they are not really hidden variable
here to discover.

\subsection{Randomized projections}

\label{nn_rnd} 

\paragraph{}

As we already said on this report, randomized projections gave surprisingly
good results given that it is only random. This conclusion is also
true here, as you can see on Figure \ref{fig:per_all_rm}, this projection
algorithm give as good results as PCA and ICA on a median basis. But
as you can see on Figure \ref{fig:per_rnd_nf} and \ref{fig:per_rnd_nf_3525},
the maximum results are not as good as for the other algorithms and
adding more feature does not necessary end up giving better results.
Thus, we get here better result with 20 features than with 30 as well
as 35 and 40. This is the result of being random, but the general
trend observed before is still true, the more you add features the
better the accuracy is.

This random projection are also not as easy to learn from as PCA.
The improvement of validation accuracy on the very first steps is
small, but this is negligible in comparison to the one with ICA.

We must also highlight that their is no major overfitting occurring
here, at the contrary of ICA and PCA. This show that overfitting is
not only dependent on the shape of the data and of the network but
is also highly dependent on the content of the data. So, we can assume
here that random projection did not produce features that were particularly
relevant on the train dataset but were completely uncorrelated to
the class in the test dataset. But as the results show, this come
with the cost of a lower accuracy.

\begin{figure}
\centering \begin{subfigure}[t]{0.7\columnwidth} \centering
\includegraphics[width=1.1\linewidth]{{../plotter/graphs/random_epoch_val_categorical_accuracy_number_of_feature}.png}
\caption{Effect of the number of features for all tested layers}
\label{fig:per_rnd_nf} \end{subfigure} \begin{subfigure}[t]{0.7\columnwidth}
\centering \includegraphics[width=1.1\linewidth]{{../plotter/graphs/random_epoch_val_categorical_accuracy_number_of_feature_35_25}.png}
\caption{Effect of the number of features for $(35)$ and $(25)$ layers}
\label{fig:per_rnd_nf_3525} \end{subfigure} \caption{Validation accuracy according to the epoch with random projections}
\label{fig:per_rnd} 
\end{figure}

\paragraph{}

In conclusion, random projection is really efficient when you consider
that it is only random and thus, very fast. But it did not allow as
good results as PCA and ICA, particularly when looking for the best
ones. But we must emphasize that it is particularly powerful when
you need to cut a lot of dimensions, in such case it even sometimes
manage better results than PCA and ICA.

\subsection{LDA}

\paragraph{}

LDA is completely in an other world than the previous reduction dimension
algorithms. First because of its supervised approach, in which you
say what information you want to keep. As discussed before, this strategy
is often more a problem than a solution, but here we are exactly in
the case where their are big benefit to use it. Second because of
its results, as you can see on Figure \ref{fig:per_all_rm}, LDA is
by far the best performer here and get result as good as the one get
during homework 1, with the full dataset.

The first thing you probably notice on Figure \ref{fig:per_lda_nf}
is that, apart from always failing 5 feature case, no matter the number
of feature you add, you only gain very little accuracy. The only variation
occurring here is due to the different layers, but as in the previous
case no layers take more advantage than another of this particular
data. Thus we again fund $(35)$ and $(25)$ as the best performer.
From this result we can things that this dataset is can probably be
condensed in about ten dimensions and still contain enough information
to find the original player, with the same accuracy than with the
72 original dimensions. But that does not say that we did not loss
any information, we just kept the ones pertinent in that case. LDA
is the only case were the perceptron was able to reach the 100\% accuracy
on training set with as few as 10 features. If one can consider that
going there only result in overfit, this show that those dimensions
has all the informations needed.

It is this information on labels that give LDA such a pertinence here.
As every supervised algorithm, LDA can overfit and loss its generalization
capabilities. Here we trained both LDA and the perceptron on a dataset
and computed accuracy on another one, and so we are pleased to see
that the transformation learn by LDA is still pertinent on our test
dataset.

As we stated before, thanks to LDA, the perceptron quickly achieve
an accuracy of 100\% on training data. It is then not very surprising
to see the NN overfit, but this overfitting is quite slower than the
one seen in PCA or ICA. Probably because the dimensions chosen by
LDA are far more generalizable to the test dataset.

In addition, in terms of epoch, the training time of the perceptron
is equivalent to the one get for PCA. This timing is also similar
to the one get in homework one, and we can suppose that it will be
hard to be faster than that while reducing the number of dimensions.

\begin{figure}
\centering \begin{subfigure}[t]{0.7\columnwidth} \centering
\includegraphics[width=1.1\linewidth]{{../plotter/graphs/LDA_epoch_val_categorical_accuracy_number_of_feature}.png}
\caption{Effect of the number of features for all tested layers}
\label{fig:per_lda_nf} \end{subfigure} \begin{subfigure}[t]{0.7\columnwidth}
\centering \includegraphics[width=1.1\linewidth]{{../plotter/graphs/LDA_epoch_val_categorical_accuracy_number_of_feature_35_25}.png}
\caption{Effect of the number of features for $(35)$ and $(25)$ layers}
\label{fig:per_lda_nf_3525} \end{subfigure} \caption{Validation accuracy according to the epoch with random projections}
\label{fig:per_lda} 
\end{figure}

\paragraph{}

We must admit that we mostly chosen LDA in this report for this particular
section and were pleased by such good results. LDA seams to perfectly
fit to its task here, it allow to quickly reduce the number of dimensions
to give to a learning method, which can help to improve its results.
Depending on its structure a perceptron can be do feature reduction
in an internal layer, but this dimension reduction is not as efficient
as the one we have shown here. We never managed to get a perceptron
to have anything like similar to the results we get here when it had
a layer of ten neurones. Showing the importance of dimension reduction
algorithm.

\subsection{Conclusion}

% nn depend on data not structure
% curse of dimention != nb feature

\paragraph{}

According to the results of this section, we can say that the learning
capability of a NN depend more on the data than on the internal structure.
If we have shown in homework 1 that some choice of architecture can
completely destroy the accuracy, the size of the layers does not have
as much importance as the information available in the data.

We can also add that the curse of dimensions is certainly linked with
the number of input features to a NN, but as we saw with ICA, the
hidden dimensions of the information also has its importance. Particularly
we showed that it is hard for a NN to understand that a feature is
just noise and has no relation with a label.

%%%%%%%%%%%%%%%%%%%%%%%%%%%%%%%%%%%%%%%%%%%%%%%%%%%%%%%%%%%%%%%%%%%%%%%%%%%%%%%%%%%%%%%%%%%%%%%%%

\section{Adding cluster as labels}

In this last section we will study the effect of adding the results
of the clustering algorithm as new features to input into the perceptron.
It is hard in our case to consider clustering as a dimension reduction
algorithm as we have 200 classes for a maximum of 72 original feature,
still adding this information can provide additional information to
the perceptron, particularly if some cluster are highly correlated
to a given class. In the same time adding 200 features will probably
not help the NN to find the best way to do such links. In addition
the results of section \ref{dr} make us doubt such an improvement,
as clustering do not look pertinent on our experiment.

As a global trend, you can see on Figure \ref{fig:clusper_all}, that
the best results are significantly lower than one the previous section,
with a maximum accuracy around 70\% while it was above 80\% in section
\ref{nn}, thanks to LDA. Another major change is that every reduction
method give very similar results, on a median value perspective as
well as on a maximal value.

We must also note on Figure \ref{fig:clusper_all_cm}, that while
we preferred the results of EM over the one of K-Means in section
\ref{dr}, here K-Means allow better results than EM no matter the
metric you use. Which is probably a good news as K-Means run much
faster than EM.

\begin{figure}
\centering \begin{subfigure}[t]{0.7\columnwidth} \centering
\includegraphics[width=1.1\linewidth]{{../plotter/graphs/clusper_all_epoch_val_categorical_accuracy_reduction_method}.png}
\caption{Effect of the reduction method on validation accuracy}
\label{fig:clusper_all_rm} \end{subfigure} \begin{subfigure}[t]{0.7\columnwidth}
\centering \includegraphics[width=1.1\linewidth]{{../plotter/graphs/clusper_all_epoch_val_categorical_accuracy_clustering_method}.png}
\caption{Effect of the clustering method on validation accuracy}
\label{fig:clusper_all_cm} \end{subfigure} \caption{Validation accuracy according to the epoch When cluster are added
as new features}
\label{fig:clusper_all} 
\end{figure}

On Figure \ref{fig:clusper_all_rm}, you can note that in a median
basis, all the reduction method have difficulties to start learning
something useful, while in previous section this case only show up
for ICA. At the same time the maximum show that at least a set of
parameters do not follow this rule. The parameter responsible for
such a change was easy to find, its effect are displayed of Figure
\ref{fig:clusper_all_layers}. As you can see the layers are by far
the most important parameters here, with the one connecting directly
input to output neurones getting the best results. This is not surprising
as all the others try to reduce the number of dimension in their intern
layers. We must also stress that $()$ is the layer which has the
more parameters, with ten times more of them than training examples,
explaining the overfitting. Even if, it appear to be stronger in the
others layers, on which it probably occur because of their dimension
reduction component, forcing the NN to chose some input to weight
more than others.

\begin{figure}
\centering \includegraphics[width=0.7\linewidth]{{../plotter/graphs/clusper_all_epoch_val_categorical_accuracy_layers}.png}
\caption{Effect of the clustering method on validation accuracy for different
layers}
\label{fig:clusper_all_layers} 
\end{figure}

\subsection{PCA}

\paragraph{}

PCA follow here the general trend, in such a way that we can qualify
it as the median dimension reduction algorithm here. Thus it follow
all general observation discussed above and give only little advantage
to K-Means over EM compared to most of the other algorithms.

\begin{figure}[h!]
\centering \includegraphics[width=0.7\linewidth]{{../plotter/graphs/clusper_PCA_epoch_val_categorical_accuracy_clustering_method}.png}
\caption{Effect of the clustering method on validation accuracy when using
PCA}
\label{fig:clusper_pca_cm} 
\end{figure}

\subsection{ICA}

\paragraph{}

ICA is again very particular. As show on Figure \ref{fig:clusper_ica_cm},
it is the only case where both K-Means and EM have very similar results,
even if K-Means is still a little above EM. ICA is also the only case
where layer $()$ do not manage to converge on the first epochs. This
probably come from irrelevant clusters as we observed on section \ref{dr_ica}
or simply because the features are difficult to understand as discussed
on section \ref{nn_ica}.

\begin{figure}[h!]
\centering \includegraphics[width=0.7\linewidth]{{../plotter/graphs/clusper_ICA_epoch_val_categorical_accuracy_clustering_method}.png}
\caption{Effect of the clustering method on validation accuracy when using
ICA}
\label{fig:clusper_ica_cm} 
\end{figure}

\subsection{Randomized Projections}

\paragraph{}

Random projections is the most discriminant algorithm we have here
when it come to compare EM and K-Means. While with EM the perceptron
still does better than chance, the median result show that it predict
the wrong class 2 times over 3. Which must be compared to the results
get on section \ref{nn_rnd}, where the results were not the best,
but still far better than this case, which only has more informations
as input. On a K-Means point of view, the results are very similar
to the one of the other algorithms, showing once again that random
projection may look stupid but is still very efficient particularly
with regard to its complexity.

\begin{figure}[h!]
\centering \includegraphics[width=0.7\linewidth]{{../plotter/graphs/clusper_random_epoch_val_categorical_accuracy_clustering_method}.png}
\caption{Effect of the clustering method on validation accuracy when using
random projections}
\label{fig:clusper_rnd_cm} 
\end{figure}

\subsection{LDA}

\paragraph{}

The surprising fact with LDA is the difference between K-means and
EM, which is close to be as important as the one observed for random
projection. This is again very surprising given that without this
additional cluster information the perceptron manage to get great
results. We suppose that this huge difference is because LDA try to
split the data in a K-Means pertinent way that do not take into account
the probabilistic vision of EM.

Another important fact is that again this new information confuse
more than help the NN.

\begin{figure}[h!]
\centering \includegraphics[width=0.7\linewidth]{{../plotter/graphs/clusper_LDA_epoch_val_categorical_accuracy_clustering_method}.png}
\caption{Effect of the clustering method on validation accuracy when using
LDA}
\label{fig:clusper_lda_cm} 
\end{figure}

\subsection{Conclusion}

% what will have been the results with the class as input ?
% class are probably right in some point but not enough to allow generalisation
% random forest ?

\paragraph{}

We can thus say that adding the result of a clustering algorithm as
features for a neural network does not help it to get better results,
at least in our specific case. An experience that would have been
interesting here, is to replace the computed cluster by the wanted
classes to see if the problem come from the structure of the perceptron
or from the clusters.

One explanation of the results we get here is that the cluster generated
on the train dataset give good insight to find the desired class,
which are used by the perceptron, but that those cluster does not
extend to the test data set and are in a way an overfit of the train
data. This conclusion is strengthened by the fact that accuracy on
the train data set converge far quicker toward perfect results on
this section than on section \ref{nn}. So, we assume that the perceptron
take advantage here to learn more efficiently, but that it ether overfit
on the train data or the cluster information is not coherent from
a dataset to the other.

It would have been interesting to also explore other machine learning
methods here as for examples random forest. Which are known to be
more efficient with large number of features, in addition to show
very good results on the original dataset.

%%%%%%%%%%%%%%%%%%%%%%%%%%%%%%%%%%%%%%%%%%%%%%%%%%%%%%%%%%%%%%%%%%%%%%%%%%%%%%%%%%%%%%%%%%%%%%%%%

\section{Conclusion}

\paragraph{}

This work allowed us to discover the advantages and drawback of dimension
reduction and clustering algorithms, as well as get a first experience
with them.

The first section showed us that it is very difficult to interpret
clustering results, even when you know what you are looking for. This
interpretation, mostly rely on graphical representation which is difficult
to do in more than three dimensions. Interpret data with more than
four dimensions can be very difficult, especially when you do not
know what you are looking for, as in data mining.

We learned, then, that dimension reduction can be a very powerful
tool when used in the right range. The difficulty being to find that
range, with the same problem of visualization stated previously.

In section \ref{nn}, we explored the possibility given by the dimension
reduction to train a neural network. Study which allowed us to learn
that even if in a theoretical point of view it is better to have fewer
dimensions as input, loss of information in the input is often worse
in term of validation accuracy. This exploration also show us, once
more, that the representation used for the data given to a neural
network is very important and can make it very difficult for a NN
to learn something.

We finally discovered that clustering on our dataset was not stable
enough to generate additional features to train our perceptron. But
learned that such approach can be useful in some situations.

\paragraph{}

Dimensions reduction was something we have already heard of, but not
properly use before this semester. It was very interesting for us
to learn more about them, in particular to experiment with them, which
will probably allow us to use them when needed.

This is far less true for clustering algorithm, as we already had
some exposition to it previously. If this assignment can be considered
a bit light on the cluster side, it completely fit with to the goal
of the courses to our mind. As we had less to learn on that side,
we may have gone over this subject a bit too quickly. If so, please
accept our apologies.

Our goal here was not to find the best parameters possible to have
good results. While we still tried to chose coherent parameters, we
mainly wanted to study the effects of different parameters for our
problems. In order to learn how the different algorithm react to them.
In the goal to be able to do correct choice when we will have to use
similar algorithm in the future. If this was not the goal of this
homework, it was at least the one of our work.
\end{document}
